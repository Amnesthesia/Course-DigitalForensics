\section{Survey of related material}
\label{sec:related}
When researching the topic of this article, we looked into several known and previously covered issues with the use of TOR and anonymity.

One particular technology that has recieved quite a bit attentio the last year is "Canvas fingerprinting", a technology meant to replace cookies in tracking a unique user across websites. It functions by instructing the browser of the user to draw a figure, using html5, the variations in browser, GPU, drivers and other settings and specifications. It is possible to identify a user to some degree, while the previously mentioned settings are not always unique for every user the technology has some shortcommings.~\cite{wiki_canvas}

Somewhat related to this is the standard way of tracking unique users, cookies.
It exists many different types of cookies, but the the short summary of it is that its a very common way to track users across one or several websites. It has some shortcommings, due it usually being limited to the current browser/user/system.~\cite{wiki_cookie}

%uuid2; This cookie contains a unique randomly-generated value that enables the Platform to distinguish browsers and devices. It is matched against information – such as advertising interest segments and histories of ads shown in the browser or device – provided by clients or other third parties and stored on the Platform.

%key: torrouter
%key: fsecapt
%key: nsamitm
%key: wiki_canvas
%key: appnexus
%key wiki_cookie
%~\cite{key}


\begin{itemize}
\item browser canvasing?
\item stuff
\end{itemize}