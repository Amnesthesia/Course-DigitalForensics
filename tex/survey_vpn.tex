\subsection{VPN}
% ~\cite{wiki_vpn}
According to wikiepedia, virtual private network (VPN)
\begin{quote}
extends a private network across a public network, such as the Internet. It enables a computer or Wi-Fi-enabled device to send and receive data across shared or public networks as if it were directly connected to the private network, while benefiting from the functionality, security and management policies of the private network
\end{quote}
This means that a vpn solution can assist in maintaining the confidentiality and integrity of your communication.
There are a few different types of VPNs - OpenVPN, IPSec and PPTP. They all have their pros and cons (although I personally prefer OpenVPN), and the various types should be looked into to  the needs of yourself or your organization.