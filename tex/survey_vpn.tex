\subsection{VPN}
 
According to wikipedia~\cite{wiki_vpn}, virtual private network (VPN):
\begin{quote}
extends a private network across a public network, such as the Internet. It enables a computer or Wi-Fi-enabled device to send and receive data across shared or public networks as if it were directly connected to the private network, while benefiting from the functionality, security and management policies of the private network
\end{quote}
This means that a vpn solution can assist in maintaining the confidentiality or integrity of your communication.
This also means that a malicious agent could use a VPN to mask his location or identity from a investigator. Some providers of VPN-services advertize as not logging anything. Combining this fact with the fact that a malicious agent could choose to use a VPN-provider from a country that does not cooperate with the law-enforcement of the investigators country and gaining meaningful evidence could prove difficult.
However, depending on what kind of information is logged by a victim(and if this information has its integrity intact), the investigating party may find information that can identify the perpetrator to some extent, in cases where cookies/browser canvasing was successfully used and logged.


There are a few different types of VPNs - OpenVPN, IPSec and PPTP. They all have their pros and cons , and the various types should be looked into to  the needs of yourself or your organization.
OpenVPN~\cite{open_vpn_wiki} is, as the name suggests, an open-source solution.