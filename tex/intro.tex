\section{Introduction}
For our project we chose proposal number 2: Experiment or theoretical analysis of cyber crime using VPN, TOR network or proxys for anti-forensics

\begin{quote}
Anti-forensics methods are heavily used to avoid that digital investigators to identify user committing cyber-crime, e.g., economic fraud, piracy (sharing/downloading torrents), or child pornagraphy. Digital forensics investigation is challenging when these techniques are used with regards to legal aspects and analysis techniques.
\end{quote}


In this project we have looked at what kind of information it is possible to acquire from hosting your own TOR exit node / VPN server. \\

Most TOR exit nodes are hosted in a different country than your own, and some of these servers might cost a noticable ammount of money to keep up, due to the network traffic going through them. It is therefore interesting for us to know what kind of information the administrator of these servers can see and to what extent they can identify/fingerprint a user.

%kan droppe nedenstående?
To summarize:
we have looked at what kind of information is visible such as usernames, e-mail adresses, user-agents, websites visited and much more..
%shoudl probably chagne the wording here, "and much more"... 
